\documentclass{book}

\usepackage[a4paper]{geometry}
\usepackage{hyperref}
\usepackage{listings}
\usepackage{color}
\usepackage{bibentry}
\nobibliography*

\title{Effective Qt}
\author{Bart Vandewoestyne}

\newcommand{\file}[1]{\texttt{#1}}
\newcommand{\code}[1]{\texttt{#1}}

\newcounter{guideline}
\newenvironment{guideline}[1][]{\refstepcounter{guideline}\par\medskip
   \noindent \textbf{Guideline~\theguideline. #1} \em \rmfamily}{\medskip}

\begin{document}

\lstset{language=C++,
        backgroundcolor=\color{white},
        basicstyle=\ttfamily,
        showspaces=false,
        showstringspaces=false,
        frame=shadowbox,
        rulesepcolor=\color{blue}}

% TODO: book cover from https://www.flickr.com/photos/86530412@N02/7987532186 (contact author)

\maketitle

\tableofcontents

\section{QObject}

\begin{guideline}
Prefer to use the \code{Q\_OBJECT} macro for subclasses of \code{QObject}.
\end{guideline}

Do this regardless of whether or not your subclasses of \code{QObject} actually use signals, slots, and properties.
References:
\begin{itemize}
\item \bibentry{metaobjectsystem}
\end{itemize}

\section{Signals and Slots}
\begin{guideline}
From Qt 5.8 on, always define \code{QT\_NO\_NARROWING\_CONVERSIONS\_IN\_CONNECT} in your \file{.pro} file.
\end{guideline}
References:
\begin{itemize}
	\item \url{https://www.kdab.com/disabling-narrowing-conversions-in-signal-slot-connections/} (TODO: add to bibtex file).
\end{itemize}

\section{QString}

\begin{guideline}
Use the multi-argument form of \code{QString::arg()}.
\end{guideline}

Instead of
\begin{lstlisting}
QString("%1 %2 %3").arg(1).arg(2).arg(3)
\end{lstlisting}
use
\begin{lstlisting}
QString("%1 %2 %3").arg(QString::number(1),
                        QString::number(2),
                        QString::number(3));
\end{lstlisting}
References:
\begin{itemize}
\item \bibentry{qstring}
\item \bibentry{mutz2015}
\end{itemize}

\section{Containers}

\begin{guideline}
Remember that \lstinline|QList| has nothing to do with \lstinline|std::list|, and \code{QSet} has nothing to do with \code{std::set}.
\end{guideline}
References:
\begin{itemize}
\item \bibentry{mutz2011}
\end{itemize}

\begin{guideline}
Be aware of the two Qt container APIs, Qt-ish and STL-compatible, and avoid mixing their use in the same project.
\end{guideline}
If in doubt, go with the STL API — it won’t change come Qt 5.If in doubt, go with the STL API — it won’t change come Qt 5.
References:
\begin{itemize}
\item \bibentry{mutz2011}
\end{itemize}

\section{Dynamic casting}

\begin{guideline}
Prefer \code{qobject\_cast()} over \code{dynamic\_cast()} for QObject classes.
\end{guideline}

The \code{qobject\_cast()} function behaves similarly to the standard C++ \code{dynamic\_cast()}, with the advantages that it doesn't require RTTI support and it works across dynamic library boundaries.

\section{Coding practice}

\begin{guideline}
Use \code{QT\_FATAL\_WARNINGS}
\end{guideline}
Set the \code{QT\_FATAL\_WARNINGS} environment variable so that \code{qWarning()} exits after printing the warning message.

\begin{guideline}
% https://www.semipol.de/2010/08/09/unused-parameters-in-cpp.html
% http://stackoverflow.com/questions/5765320/clean-way-to-eliminate-unused-parameter-widget-warning-generated-by-qgraphic
Don't use the \lstinline{Q_UNUSED} macro to indicate that a parameter is not used, but simply comment the variable name in the function parameter list.
\end{guideline}
The reason for this is that it has the drawback of having a potential for confusion.  Maybe sometime later you need the parameter but forget to remove the \lstinline{Q_UNUSED} macro, hence generating something like this:
\begin{lstlisting}
foo(int aParameter) {
    Q_UNUSED(aParameter)
    // 30 lines of other method code
    int anotherVariable = 30 * aParameter;
}
\end{lstlisting}
Suddenly your code documents that a variable isn't used but actually it is and the compiler can't detect this error.  Instead, you can simply comment the variable name:
\begin{lstlisting}
foo(int /*aParameter*/) {
    // 30 lines of other method code
    int anotherVariable = 30 * aParameter;
}
\end{lstlisting}
Now the warning is gone, too and the compiler can detect either the illegal use of the variable in line three or the illegal statement that this variable is unused by purpose.
Another advantage of avoiding the \lstinline{Q_UNUSED} macro is that it makes your code less dependent of the Qt libraries.

\bibliographystyle{plain}
\bibliography{references}

\end{document}
