\documentclass{book}

\usepackage[a4paper]{geometry}
\usepackage{hyperref}

\title{Effective Qt}
\author{Bart Vandewoestyne}

\newcommand{\file}[1]{\texttt{#1}}
\newcommand{\code}[1]{\texttt{#1}}

\newenvironment{guideline}
{ \begin{quote} }
{ \end{quote} }

\begin{document}

\maketitle

\tableofcontents

\section{QObject}

\begin{guideline}
Prefer to use the \code{Q\_OBJECT} macro for subclasses of \code{QObject}.
\end{guideline}
Do this regardless of whether or not your subclasses of \code{QObject} actually use signals, slots, and properties.

\section{QString}

\begin{guideline}
Use the multi-argument for of \code{QString::arg()}.
\end{guideline}
Instead of
\begin{verbatim}
QString("%1 %2 %3).arg(1).arg(2).arg(3)
\end{verbatim}
use
\begin{verbatim}
QString("%1 %2 %3").arg(QString::number(1), QString::number(2), QString::number(3));
\end{verbatim}
(See \url{http://doc.qt.io/qt-4.8/qstring.html#arg-2} and \url{https://youtu.be/Ov7s0GgBbOQ?t=45m40s}.

\section{Dynamic casting}

\begin{guideline}
Prefer \code{qobject\_cast()} over \code{dynamic\_cast()} for QObject classes.
\end{guideline}
The \code{qobject\_cast()} function behaves similarly to the standard C++ \code{dynamic\_cast()}, with the advantages that it doesn't require RTTI support and it works across dynamic library boundaries.

\section{Coding practice}

\begin{guideline}
Use \code{QT\_FATAL\_WARNINGS}
\end{guideline}
Set the \code{QT\_FATAL\_WARNINGS} environment variable so that \code{qWarning()} exits after printing the warning message.

\end{document}
