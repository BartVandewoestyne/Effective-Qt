\documentclass{book}

\usepackage[a4paper]{geometry}
\usepackage{hyperref}
\usepackage{listings}
\usepackage{color}
\usepackage{bibentry}
\nobibliography*

\title{Effective Qt}
\author{Bart Vandewoestyne}

\newcommand{\file}[1]{\texttt{#1}}
\newcommand{\code}[1]{\texttt{#1}}

\newcounter{guideline}
\newenvironment{guideline}[1][]{\refstepcounter{guideline}\par\medskip
   \noindent \textbf{Guideline~\theguideline. #1} \em \rmfamily}{\medskip}

\begin{document}

\lstset{language=C++,
        backgroundcolor=\color{white},
        basicstyle=\ttfamily,
        showspaces=false,
        showstringspaces=false,
        frame=shadowbox,
        rulesepcolor=\color{blue}}

% TODO: book cover from https://www.flickr.com/photos/86530412@N02/7987532186 (contact author)

\maketitle

\tableofcontents

\section{QObject}

\begin{guideline}
Prefer to use the \code{Q\_OBJECT} macro for subclasses of \code{QObject}.
\end{guideline}

Do this regardless of whether or not your subclasses of \code{QObject} actually use signals, slots, and properties.

\section{QString}

\begin{guideline}
Use the multi-argument form of \code{QString::arg()}.
\end{guideline}

Instead of
\begin{lstlisting}
QString("%1 %2 %3").arg(1).arg(2).arg(3)
\end{lstlisting}
use
\begin{lstlisting}
QString("%1 %2 %3").arg(QString::number(1),
                        QString::number(2),
                        QString::number(3));
\end{lstlisting}
References:
\begin{itemize}
\item \bibentry{qstring}
\item \bibentry{mutz2015}
\end{itemize}

\section{Containers}

\begin{guideline}
Remember that \lstinline|QList| has nothing to do with \lstinline|std::list|, and \code{QSet} has nothing to do with \code{std::set}.
\end{guideline}
References:
\begin{itemize}
\item \bibentry{mutz2011}
\end{itemize}

\begin{guideline}
Be aware of the two Qt container APIs, Qt-ish and STL-compatible, and avoid mixing their use in the same project.
\end{guideline}
If in doubt, go with the STL API — it won’t change come Qt 5.If in doubt, go with the STL API — it won’t change come Qt 5.
References:
\begin{itemize}
\item \bibentry{mutz2011}
\end{itemize}

\section{Dynamic casting}

\begin{guideline}
Prefer \code{qobject\_cast()} over \code{dynamic\_cast()} for QObject classes.
\end{guideline}

The \code{qobject\_cast()} function behaves similarly to the standard C++ \code{dynamic\_cast()}, with the advantages that it doesn't require RTTI support and it works across dynamic library boundaries.

\section{Coding practice}

\begin{guideline}
Use \code{QT\_FATAL\_WARNINGS}
\end{guideline}

Set the \code{QT\_FATAL\_WARNINGS} environment variable so that \code{qWarning()} exits after printing the warning message.

\bibliographystyle{plain}
\bibliography{references}

\end{document}
